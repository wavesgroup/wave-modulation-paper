\documentclass[draft]{agujournal2019}
\usepackage{amsmath}
\usepackage{amssymb}
\usepackage{url} %this package should fix any errors with URLs in refs.
\usepackage{lineno}
\usepackage{soul}
\linenumbers

\draftfalse

\journalname{JGR: Oceans}

\begin{document}

\title{Dispersion of short waves riding on longer waves}

\authors{Milan Curcic\affil{1,2}}

\affiliation{1}{
  Rosenstiel School of Marine, Atmospheric, and Earth Science,
  University of Miami,
  Miami, FL
}
\affiliation{2}{
  Frost Institute for Data Science and Computing,
  University of Miami,
  Miami, FL
}

\correspondingauthor{Milan Curcic}{mcurcic@miami.edu}

\begin{keypoints}
\item We revisit analytical and numerical solutions for hydrodynamic modulation of short waves by longer waves from linear wave theory.
\item We extend the original solution by \citeA{longuet1960changes} to a higher-order solution.
\item In contrast to earlier studies, we find that there exist stable numerical solutions for modulation by realistic long-wave groups.
\end{keypoints}

\begin{abstract}
Hydrodynamic modulation of short ocean surface waves by longer ambient waves is
a well-known ocean surface process that pertains to remote sensing, the
interpretation of in situ wave measurements, and numerical wave forecasting.
In this paper we revisit the early analytical work by \citeA{longuet1960changes},
as well as the more recent numerical simulations by \citeA{peureux2021unsteady},
to re-evaluate the changes in short-wave properties due to the presence of longer waves.
We present two distinct improvements to the current state of the art:
First, we derive a higher-order, non-linear solution for the short-wave wavenumber
and action density as function of the long-wave phase and steepness.
This solution suggests an increase in short-wave wavenumber by 18 and 38\% for
long-wave steepenesses of 0.1 and 0.2, respectively, compared to the original
linear solution.
It is also in close agreement with the full numerical solutions up to the
long-wave steepness of 0.25.
Second, we find that the full numerical solutions, which were previously found
to be unstable for homogeneous initial conditions, are in fact stable for more
realistic long-wave groups.
The key physical interpretation from these results is that the short waves can
indefinitely grow and steepen only in the presence of inifinite long-wave trains,
an unlikely occurence in the ocean.
\end{abstract}

\section*{Plain Language Summary}
The ocean surface is typically covered by waves of many wavelengths.
Longer waves bunch up and dilate the shorter waves, and this process is
important to consider for both the measurements and the forecasting of ocean
waves.
Although there have been analytical and numerical solutions for how the short
waves should change in the presence of longer waves, they have been simplified
and incomplete.
This paper revisits and extends these solutions to propose a more accurate
description of how the short waves should change in the presence of longer
waves.

\section{Introduction}

Short ocean surface waves are modulated by longer swell.
Following the result by \citeA{unna1941white,unna1942waves,unna1947sea},
\citeA{longuet1960changes} introduced an analytical solution for hydrodynamic
modulation of short waves at the crest of an underlying longer wave:

\begin{equation}
\label{eq:tau}
\widetilde{k} = k (1 + a_L k_L)
\end{equation}
where $\widetilde{k}$ and $k$ are the modulated and unmodulated wavenumbers of
short waves, respectively, and $a_L$ and $k_L$ are amplitude and wavenumber
of the long wave, respectively.
This result has been influential on the development of remote sensing of ocean
surface waves and currents \cite{keller1975microwave,hara1994hydrodynamic}
and has been used even recently \cite{peureux2021unsteady} to simulate
unsteady growth of short-waves due to longer waves.

TODO extend with background.

In this study, we extend the \citeA{longuet1960changes} theory (hereafter L-HS)
to a higher-order solution following the linear wave theory, and present revised
phase-averaged estimates of short-wave properties in presence of long waves. 

\section{Governing equations}

In this paper we consider inviscid, irrotational, and incompressible flow.
In these conditions the linear (sine-like), deep-water, surface gravity waves
obey the dispersion relation:

\begin{equation}
\label{eq:dispersion}
\omega = \sqrt{gk} + k U
\end{equation}
where $\omega$ is the angular frequency, $g$ is the gravitational acceleration,
$k$ is the wavenumber, and $U$ is the mean advective current in the direction
of the wave propagation.
A current in the direction of the waves thus increases its absolution frequency,
wavenumber remaining unchanged.
An important limitation here is that $U$ must be slow-varying on the scales of
the wave period \cite{bretherton1968wavetrains}.
In the context of short waves riding on long waves, the mean advective current
is the horizontal near-surface orbital velocity of the long wave.

The second important relation is the conservation of wave crests:

\begin{equation}
\label{eq:wave_crests}
\dfrac{\partial k}{\partial t}
+ \dfrac{\partial \omega}{\partial x}
= 0
\end{equation}
where $t$ and $x$ are time and space, respectively.
This relation states that the wavenumber must change when the frequency varies
in space.

Equations (\ref{eq:dispersion}) and (\ref{eq:wave_crests}) only describe the
evolution of the short-wave wavenumber.
To describe the evolution of the wave amplitude, we use the conservation of wave
action in absence of sources (e.g. due to wind) and sinks (e.g. due to whitecapping):

\begin{equation}
\label{eq:wave_action}
\dfrac{\partial N}{\partial t}
+ \dfrac{\partial}{\partial x} \left[\left(C_g + U\right)N\right]
= 0
\end{equation}
where $N$ is the action density, and $C_g = \partial \omega / \partial k$ is the
group speed.
In deep water, the group speed is half the phase speed, so $C_g = 1/2\sqrt{g/k}$.

As the short waves ride on the surface of the long wave,
they move in an accelerated reference frame and will thus experience effective
gravitational acceleration that changes in both space and time \cite{longuet1986eulerian,longuet1987propagation}.
Inserting Eq. (\ref{eq:dispersion}) into (\ref{eq:wave_action}) we obtain:

\begin{equation}
\label{eq:wavenumber}
\dfrac{\partial k}{\partial t}
+ \left(C_g + U\right) \dfrac{\partial k}{\partial x}
+ k \dfrac{\partial U}{\partial x}
+ \dfrac{1}{2} \sqrt{\dfrac{k}{g}} \dfrac{\partial g}{\partial x}
= 0
\end{equation}
From left to right, the terms in this equation represent the change in wavenumber
due to the propagation and advection by ambient current $U$, the convergence of
the ambient current, and the spatial inhomogeneity of the gravitational acceleration.
In a non-accelerated reference frame (i.e. in absence of longer waves), the last
term is zero.
However, if gravitational acceleration varies in space, this term is necessary
to conserve the wave crests.

Similar to Eq. \ref{eq:wavenumber}, we expand Eq. \ref{eq:wave_action} to obtain:

\begin{equation}
\label{eq:wave_action2}
\dfrac{\partial N}{\partial t}
+ \left(C_g + U\right) \dfrac{\partial N}{\partial x}
+ N \dfrac{\partial U}{\partial x}
+ N \dfrac{\partial C_g}{\partial x}
= 0
\end{equation}
This equation is similar to Eq. \ref{eq:wavenumber}, except for the last term
which here represents the inhomogeneity of the group speed.
As \citeA{peureux2021unsteady} explained, this term is responsible for unstable
growth of wave action in infinite long-wave trains.
Eqs. \ref{eq:wavenumber} and \ref{eq:wave_action2} will be the governing equations
that we solve numerically in section \ref{section:numerical_solutions}.

For the ambient forcing, we consider a train of monochromatic long wave
defined, to the first order in steepness, by the elevation $\eta_L = a_L cos{\psi}$,
where $a_L$ is the long-wave amplitude, $\psi = k_L x - \omega_L t$ is the
long-wave phase, and $\omega_L$ its angular frequency.

At this time we define the long-wave steepness $\varepsilon_L = a_L k_L$ which
we will use thoughout the rest of the paper as a scaling parameter.
The long wave induces horizontal and vertical orbital velocities whose values
at the surface $\eta_L$ are:

\begin{equation}
\label{eq:U_L}
U = a_L \omega_L e^{\varepsilon_L \cos{\psi}} \cos{\psi}
\end{equation}

\begin{equation}
\label{eq:w_L}
W = a_L \omega_L e^{\varepsilon_L \cos{\psi}} \sin{\psi}
\end{equation}

Eqs. \ref{eq:dispersion}-\ref{eq:w_L} is the full system of equations that we
exploit in the remainder of this paper.

\section{Analytical solutions}
\label{section:analytical_solutions}

\subsection{Wavenumber and amplitude modulation}
\label{subsection:analytical_solutions}

Solving analytically for the wavenumber requires dropping spatial gradients of
$k$ in Eq. \ref{eq:wavenumber} and assuming homogeneous $g$:

\begin{equation}
\label{eq:wavenumber2}
\dfrac{\partial k}{\partial t}
= - k \dfrac{\partial U}{\partial x}
\end{equation}
Although \citeA{peureux2021unsteady} pointed out that the solution by L-HS
requires assuming homogeneous group speed of the short-wave field, in fact,
it also requires assuming no horizontal propagation and advection of the short
waves.
Combining Eqs. \ref{eq:U_L} and \ref{eq:wavenumber2}, and integrating it in time
yields:

\begin{equation}
\label{eq:k_short_exact}
\widetilde{k}(\psi) = k e^{\varepsilon_L \cos{\psi} e^{\varepsilon_L \cos{\psi}}}
\end{equation}

Notice that the Taylor expansion of Eq. \ref{eq:k_short_exact} to the first order
recovers the original solution by L-HS:

\begin{equation}
\label{eq:k_short_lhs}
\widetilde{k}(\psi) = k (1 + \varepsilon_L \cos{\psi})
\end{equation}

The L-HS solution requires two approximations in their solution:
First, evaluating the long-wave velocity at $z = 0$
rather than $z = \eta_L$; and second, evaluating the solution to the first order
of the Taylor expansion series.
Both of these approximations cause an underestimate of the short-wave
modulation magnitude (Fig. \ref{fig:wavenumber_modulation}).
The original L-HS solution is a cosine wave with an amplitude equal
to the long-wave steepness.
The other two solutions both introduce an increase in wave modulation
on the crest and a decrease in the trough.
At the crest, the relative modulation predicted by the exact L-HS solution
is 10.7\% larger than that predicted by the 1st order approximate L-HS
solution, while the solution that evaluates the long-wave orbital velocity
at the water surface rather than $z = 0$ is 38.4\% larger than that of the
1st order approximate L-HS solution.

\begin{figure}[h]
\label{fig:wavenumber_modulation}
\centering
\includegraphics[width=\textwidth]{figures/fig_wavenumber_modulation_2panel.png}
\caption{
  Comparison of relative short-wave modulation as function of the
  long-wave phase, for a moderately steep long wave of $\varepsilon = 0.1$.
  The dotted line shows the 1st-order approximated solution by L-HS,
  the dashed line shows the exact solution by L-HS,
  and the solid line shows the extended solution introduced in this paper.
}
\end{figure}

TODO amplitude modulation;
The modulation of short-wave amplitude can be derived in the same way by
dropping the non-linear terms in the wave action balance equation
(Eq. \ref{eq:wave_action2}) and integrating wave action in time:

\begin{equation}
\label{eq:wave_action_modulation}
\widetilde{N}(\psi) = N e^{\varepsilon_L \cos{\psi} e^{\varepsilon_L \cos{\psi}}}
\end{equation}

Since wave action is variance divided by the intrinsic frequency $\sigma$,
and variance scales with the square of the amplitude, the modulation of
amplitude is proportional to the square root of the modulation of wave action.

\begin{equation}
\label{eq:wave_amplitude_modulation}
\widetilde{a}(\psi) = a \sqrt{\dfrac{\sigma}{\widetilde{\sigma}} \dfrac{\widetilde{N}}{N}}
\end{equation}
To determine $\widetilde{\sigma}$, we will need to evaluate the modulation of
the gravitational acceleration at the surface of the long wave.

\subsection{Gravity modulation}
\label{subsection:gravity_modulation}

Long waves induce vertical orbital accelerations, so the short
waves that ride on their surface move in an accelerated reference frame and
experience effective gravitational acceleration that is lower at the crests and
higher in the troughs \cite{longuet1986eulerian,longuet1987propagation}.
A linear wave causes a modulation in gravity acceleration at its surface:

\begin{equation}
\label{eq:gravity_modulation_linear}
\widetilde{g}(\psi) = g + \dfrac{\partial W}{\partial t} = g \left[
  1 - \varepsilon_L e^{\varepsilon_L \cos{\psi}} \left(
    \cos{\psi} - \varepsilon_L \sin^2{\psi}
  \right)
\right]
\end{equation}
The first-order term in $a_L k_L$ is in phase with the long-wave elevation,
so the apparent gravity of the short waves is increased at the long-wave crests
and decreased in the troughs.
The second-order term in $a_L k_L$ is positive-definite and phase-shifted by
$\pi/2$ relative to the crests.
The second-order term thus increases the apparent gravity on the front and rear
faces of the long wave (Fig. \ref{fig:gravity_modulation}).

\begin{figure}[h]
\label{fig:gravity_modulation}
\centering
\includegraphics[width=\textwidth]{figures/fig_gravity_modulation_2panel.png}
\caption{
  Relative change in gravity of short waves due to a longer wave with steepness $a_L k_L = 0.2$,
  as a function of the long-wave phase and evaluated to the first (dashed line) and second (solid line) order in $a_L k_L$.
}
\end{figure}

We can also derive the short-wave gravitational acceleration on the surface of
a non-linear Stokes wave with the elevation (to the third order in $\varepsilon_L$):

\begin{equation}
\label{eq:eta_stokes}
\eta_{St} = a_L \left[
  \cos{\psi} +
  \dfrac{1}{2} \varepsilon_L \cos{2\psi} +
  \varepsilon_L^2 \left( \dfrac{3}{8} \cos{3\psi} - \dfrac{1}{16} \cos{\psi} \right)
\right]
\end{equation}

The corresponding gravitational acceleration at the surface of the Stokes wave
is:

\begin{equation}
\label{eq:gravity_modulation_stokes}
\widetilde{g}(\psi) =
g \left\{
  1 - \varepsilon_L e^{k \eta_{St}}
  \left[ \cos{\psi} -
    \varepsilon_L \sin{\psi} \left(
      \sin{\psi}
      + \varepsilon_L \sin{2\psi}
      - \dfrac{1}{16} \varepsilon_L^2 \sin{\psi}
      + \dfrac{9}{8} \varepsilon_L^2 \sin{3\psi}
    \right)
  \right]
\right\}
\end{equation}

With the new long-wave modulated expressions for the short-wave wavenumber
(Eq. \ref{eq:k_short_exact}) and gravity
(Eqs. \ref{eq:gravity_modulation_linear}-\ref{eq:gravity_modulation_stokes}),
we can evaluate the long-wave modulated frequency as
$\widetilde{\omega}(\psi) = \sqrt{\widetilde{g} \widetilde{k}}$.
We compare the modulated intrinsic frequency for short waves on the surface of
linear and Stokes waves with $\varepsilon_L=2$ as function of phase, and their
phase-integral statistics in Fig. \ref{fig:frequency_modulation}.

\begin{figure}[h]
\label{fig:frequency_modulation}
\centering
\includegraphics[width=\textwidth]{figures/fig_frequency_modulation_2panel.png}
\caption{
  (a) Relative change in the short-wave intrinsic frequency on the surface of a
  longer linear (black) and Stokes (red) wave of steepness $\varepsilon_L = 0.2$
  as function of long-wave phase;
  (b) Long-wave phase maximum (solid), mean (dashed), and minimum (dotted)
  relative change in the short-wave intrinsic frequency as function of long-wave
  steepness in the case of a linear (black) and Stokes (red) long wave.
}
\end{figure}

\subsection{Lagrangian averaged short-wave properties}

Stokes drift is a residual Lagrangian drift that occurs due to the wave orbits
not being closed under a propagating wave
\cite{stokes1847,kenyon1969stokes,van2018stokes}:

\begin{equation}
\label{}
u_s = a^2 \omega k e^{2kz}
\end{equation}
for a monochromatic wave in deep water.
Stokes drift has been successfully used to predict material transport on the
ocean surface \cite{rohrs2012observation,curcic2016hurricane}.
It can be derived from linear wave theory by integrating the wave orbital
velocity following material orbits \cite{phillips1966dynamics}:

\begin{equation}
\label{eq:lagrangian_average}
u_s = \overline{
  u(\psi) + 
  \zeta \dfrac{\partial u}{\partial x} +
  \xi \dfrac{\partial w}{\partial z}
}
\end{equation}
where $\zeta$ and $\xi$ are horizontal and vertical excursions of the orbits,
respectively.

To obtain an average modulation of the short-wave properties, we apply the same
Lagrangian integration approach used to obtain Stokes drift.

\begin{equation}
\label{eq:stokes_wavenumber}
\widetilde{k_s} = \overline{
  \widetilde{k}(\psi) +
  \zeta \dfrac{\partial \widetilde{k}}{\partial x} +
  \xi \dfrac{\partial \widetilde{k}}{\partial z}
}
= k \left( 1 + a_L^2 k_L^2 \right)
\end{equation}

\begin{equation}
\label{eq:stokes_gravity}
\widetilde{g_s} = \overline{
  \widetilde{g}(\psi) +
  \zeta \dfrac{\partial \widetilde{g}}{\partial x} +
  \xi \dfrac{\partial \widetilde{g}}{\partial z}
}
= g \left( 1 + \dfrac{3}{2} a_L^2 k_L^2 \right)
\end{equation}

\section{Numerical solutions}
\label{section:numerical_solutions}

We now proceed to integrate the full numerical solutions for the short-wave
wavenumber and action density in presence of long waves, and in absence of
sources or sinks of wave energy.
Equations (\ref{eq:wavenumber})-(\ref{eq:wave_action2}) are discretized using
2$^{nd}$  order central finite difference in space and integrated in time using the
4$^{th}$ order Runge-Kutta method \cite{butcher1996runge}.
The space is divided into 100 grid points.
This is effectively the same equation set and numerical configuration as that of
\citeA{peureux2021unsteady}, except that, like \citeA{longuet1960changes}, their
long-wave orbital velocities are evaluated at $z=0$ instead of $z=\eta_L$.
The key difference is that evaluating orbital velocities at $z=0$ neglects the
Stokes drift of short-wave groups.
Although not significantly consequential for the modulation results here, there
is no good reason to exclude it given that its implementation is trivial.

Another difference is the choice of the numerical scheme for spatial differences,
which in their case is the MUSCL4 scheme by Kurganov and Tadmor (2000 TODO).
Although the 2$^{nd}$ order central finite difference is not appropriate for many
numerical problems, in our case it is sufficient because it is conservative and
the fields that we compute derivatives of are smooth and continuous.
We show in the next section that our numerical results in infinite long-wave
trains are qualitatively equivalent to those of \citeA{peureux2021unsteady}.

Finally, the numerical equations here are integrated in a fixed reference frame,
rather than one that moves with the long wave phase speed.
Either approach produces qualitatively equivalent results, but we find that
a fixed reference frame allows for more intuitive interpretation of the results.
However, this difference is important to keep in mind when comparing the results
of \citeA{peureux2021unsteady} and those presented here--in their Fig. 1 the
long-wave phase is fixed and the short waves are moving leftward
with the speed of $C_{pL} - C_g - U$ (neglecting any group speed inhomogeneity),
whereas in the figures that follow, the long wave is moving rightward with its
phase speed $C_{pL}$ and the short waves (that is, their energy) are moving
rightward with the speed of $C_g + U$ (neglecting any group wave inhomogeneity).

\subsection{Unstable growth in infinite long-wave trains}
\label{subsection:unstable_growth}

The key result of \citeA{peureux2021unsteady} is that of unsteady steepening of
short waves when a homogeneous field of short waves is forced by an infinite
long-wave train.
The steepening is driven by the progressive increase in the action of short
waves (as opposed to their wavenumber), and is centered on the group of short
waves that begins its journey in the trough of the long wave.
This is because the unmodulated short waves that begin in the trough enter into
the area where the surface velocities are converging.
Inversely, the short waves that begin around the crest of the long-wave will
enter into the area of surface divergence and will thus have their wave action
and wavenumber, decreased.
We reproduce this behavior in Fig. \ref{fig:modulation_3panel_infinite}, which
shows the evolution of the short-wave action, wavenumber, and steepness
modulation (here defined as the change of a quantity relative to its initial value).
After 10 long-wave periods, the short-wave action modulation $\widetilde{N}/N$
is approximately doubled, which almost exactly corresponds to the unsteady growth
reported by \citeA{peureux2021unsteady} in their Fig. 1.
Another important property of this solution is that the short-wave group that
experiences peak amplification (dampening) are not locked-in to the long-wave
crests (troughs), as they are in the L-HS solution.

\begin{figure}[h]
\label{fig:modulation_3panel_infinite}
\centering
\includegraphics[width=\textwidth]{figures/fig_modulation_3panel_infinite_wave_train.png}
\caption{
  Change in the short-wave action (left), wavenumber (middle), and steepness (right)
  relative to their initial values as function of initial long-wave phase and time.
  $k_L = 1$, $k = 10$, $\varepsilon_L = 0.1$.
}
\end{figure}

Why does this unsteady growth of short-wave action occur?
As \citeA{peureux2021unsteady} explain, "the additional change in advection
velocity due to the modulation of the short wave group speed gives an additional
amplification."
That is, it is the inhomogeneity term $N \dfrac{\partial C_g}{\partial x}$ that
is responsible for the instability.
However, the instability only occurs if the short waves are initialized as a
uniform field.
If they are initialized instead as a periodic function of the long-wave phase
such that their wavenumber (and optionally, action) is higher on the crests
and lower in the troughs, akin to the L-HS solution, the instability does not
occur.

\begin{figure}[h]
\label{fig:modulation_3panel_ramp}
\centering
\includegraphics[width=\textwidth]{figures/fig_modulation_3panel_linear_ramp.png}
\caption{
  Same as Fig. \ref{fig:modulation_3panel_infinite} but with a linear ramp
  applied to $a_L$ during the initial five long-wave periods.
}
\end{figure}

Why is it, then, that the solution is so different depending on the short-wave
initial conditions?
\citeA{peureux2021unsteady} did not provide an explanation for this behavior.
At first, it may seem that there is something special about the uniform initial
conditions that makes its solutions grow unsteadily and indefinitely.
\citeA{peureux2021unsteady} describe this scenario as "...the sudden appearance
of a long wave perturbation in the middle of a homogeneous short wave field".
The sudden appearance of long waves may be a clue.
The exact numerical simulation as shown in Fig. \ref{fig:modulation_3panel_infinite},
except for a linear ramp applied to $a_L$ (and consequently, $U$ and $W$ as well)
for the first five periods reveals that the unsteady growh no longer occurs
(Fig. \ref{fig:modulation_3panel_ramp}).

\begin{figure}[h]
  \label{fig:modulation_3panel_ramp}
  \centering
  \includegraphics[width=\textwidth]{figures/fig_modulation_ramp_timeseries.png}
  \caption{
    Maximum short-wave steepness modulation as function of time in case of
    infinite long-wave train case (blue) and the same case but with a linear
    ramp during the initial five long-wave periods (orange). The dashed black
    line corresponds to the short-wave steepness of 0.4 at which most waves are
    expected to break.
  }
  \end{figure}

\subsection{Short-wave modulation by long-wave groups}
\label{subsection:wave_groups}

\begin{figure}[h]
\label{fig:modulation_3panel_groups}
\centering
\includegraphics[width=\textwidth]{figures/fig_modulation_3panel_wave_group.png}
\caption{
  Same as Fig. \ref{fig:modulation_3panel_infinite} but for a long-wave group,
  causing the long-wave amplitude to gradually increase and peak at $a_L = 0.1$
  after five long-wave periods, and then conversely decay back to a calm sea state.
}
\end{figure}



\section{Scale analysis}
\label{section:scale_analysis}

Now that we showed that the analytical solution is consistent with full numerical
solutions for $\varepsilon_L < 0.25$, we can use analytical solutions for scale
analysis to quantify the relative contribution of tendencies in the full equation set.

TODO

\section{Conclusions}

\acknowledgments
Thanks to Peisen Tan, Nathan Laxague, and Fabrice Ardhuin for providing helpful
feedback on the early drafts of this work.

\bibliography{references.bib}

\end{document}
