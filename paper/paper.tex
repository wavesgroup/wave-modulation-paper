\documentclass[lineno]{jfm}

\usepackage{graphicx}
%\usepackage{epstopdf,epsfig}
\usepackage{newtxtext}
\usepackage{newtxmath}
\usepackage{natbib}
\usepackage{hyperref}
\hypersetup{
    colorlinks = true,
    urlcolor   = blue,
    citecolor  = black,
}
\newtheorem{lemma}{Lemma}
\newtheorem{corollary}{Corollary}
\newcommand{\RomanNumeralCaps}[1]
\linenumbers


% {\MakeUppercase{\romannumeral #1}}

\title{Revisiting the hydrodynamic wave modulation from linear wave theory}

\author{
  Milan Curcic\aff{1}
  \corresp{\email{mcurcic@miami.edu}}
}

\affiliation{
  \aff{1}Rosenstiel School of Marine, Atmospheric, and Earth Science, University of Miami, Miami, FL
  \aff{2}Frost Institute for Data Science and Computing, University of Miami, Coral Gables, FL
}

\begin{document}
\maketitle

\begin{abstract}
Hydrodynamic modulation of short ocean surface waves by longer ambient waves is
a well-known ocean surface process that affects remote sensing, the
interpretation of \textit{in situ} wave measurements, and numerical wave
forecasting.
In this paper, we revisit the linear wave theory and derive higher-order
solutions for the change of short-wave wavenumber, action density, and
gravitational acceleration due to the presence of longer waves.
We validate the analytical solutions with numerical simulations of the full wave
crest and action conservation equations.
The nonlinear analytical solutions of short-wave wavenumber, amplitude, and
steepness modulation significantly deviate from the linear analytical solutions
of \citet{longuet1960changes}, and are similar to the nonlinear numerical
solutions by \citet{zhang1990evolution}.
We also discuss the result of \citet{peureux2021unsteady} who found through
numerical simulations that the short-wave modulation grows unsteadily with
each long-wave phase.
Here we show that this unsteady growth only occurs for homogeneous initial
conditions and does not occur for more realistic long-wave groups.
Finally, the combined analytical and numerical framework allows us to quantify
the contribution of various terms in the full wave conservation equations.
The new analytical solutions are a good approximation of the full numerical
solution in long-wave steepness up to $\approx 0.2$.
\end{abstract}

\begin{keywords}
%Authors should not enter keywords on the manuscript, as these must be chosen by the author during the online submission process and will then be added during the typesetting process (see \href{https://www.cambridge.org/core/journals/journal-of-fluid-mechanics/information/list-of-keywords}{Keyword PDF} for the full list).  Other classifications will be added at the same time.
\end{keywords}

%{\bf MSC Codes }  {\it(Optional)} Please enter your MSC Codes here

\section{Introduction}

Short ocean surface waves are modulated by longer swell.
Following \citet{unna1941white,unna1942waves,unna1947sea},
\citet{longuet1960changes} introduced a steady, approximate solution for
hydrodynamic modulation of short waves by longer waves:

\begin{equation}
\label{eq:lhs1960wavenumber}
\widetilde{k} = k (1 + \varepsilon_L)
\end{equation}
where $\widetilde{k}$ and $k$ are the modulated and unmodulated short-wave wavenumbers,
respectively, and $\varepsilon_L = a_L k_L$ is the steepness of the long wave with
wavenumber $k_L$ and amplitude $a_L$.
This result influenced the development of remote sensing of ocean
surface waves and currents \citep{keller1975microwave,hara1994hydrodynamic}.
\citet{phillips1981dispersion} and \citet{longuet1987propagation} revisited this
problem by nonlinear long waves and variations in the effective gravity of short
waves, yielding significantly stronger modulation than previously predicted.
\citet{henyey1988energy} derived an analytical solution for the modulation of
short waves by long waves using Hamiltonian mechanics and reported the modulation
of similar magnitude to that of \citet{longuet1987propagation}.
\citet{zhang1990evolution} considered weakly nonlinear short waves
using a non-linear Schr\" odinger equation and found stronger wavenumber
modulation but weaker amplitude modulation than those predicted by
\citet{longuet1987propagation}.
The variety of analytical and numerical frameworks (Eulerian, Hamiltonian,
Schr\" odinger) that relatively closely reproduce steady solutions for short-wave
modulation suggests that the solutions are robust.
Alternative analytical and numerical approaches to the modulation problem remain
of interest, especially if higher degrees of nonlinearity can be considered
using simpler approaches.

More recently, \citep{peureux2021unsteady} asked whether the steady solutions
of short-wave modulation are realistic by simulating the full wave conservation
equations
They found that the short waves grow unsteadily due to the propagation of
longer waves, suggesting that the steady-state solutions may only be valid for
a few long-wave periods before the short waves begin to break due to excessive
steepness.
However, the unsteadiness of their solutions occurs in a specific scenario of
a uniform short-wave field that suddenly appears in a uniform long-wave train,
and the numerical simulations stabilize if the short-wave field is initialized
from existing steady solutions.
Nevertheless, their results do put into question whether the short-wave
modulation predicted by theory is generally steady over many long-wave periods
or not.

In this paper, we revisit this problem from linear wave theory and
derive alternative non-linear solutions for the modulation of short waves,
evaluated at the long-wave surface.
The analytical solutions yield modulations that are stronger than the original solution
by \citet{longuet1960changes}, but similar to the nonlinear numerical solutions
by \citet{longuet1987propagation} and \citet{zhang1990evolution} in long-wave
steepness up to $\approx 0.2$.
We also perform numerical simulations of the full wave crest and action
conservation laws and evaluate the effect of long-wave groups on the short waves.
The numerical simulations are used to determine the validity of the approximate
steady solution and to investigate the unsteadiness of the short-wave modulation
found by \citet{peureux2021unsteady}.
The main prior studies on hydrodynamic modulation, and their assumptions and
approaches to the solution, are summarized in Table
\ref{table:modulation_literature_summary}.

\begin{table}
\begin{center}
\def~{\hphantom{0}}
\begin{tabular}{cccccc}
Reference   & Governing & Gravity & Nonlinear  & Nonlinear   & Solution \\
            & equations & varies  & long waves & short waves &          \\
\hline
L-HS 1960   & Velocity potential  & No  & No  & No  & Analytical \\
\hline
P 1981      & Crest and action    & Yes & Yes & No  & Numerical \\
            & conservation        &     &     &     &           \\
\hline
L-H 1987    & Velocity potential  & Yes & Yes & No  & Numerical \\
\hline
HCDSW 1988  & Hamiltonian         & Yes & Yes & No  & Analytical \\
\hline
ZM 1990     & Schr\" odinger      & Yes & Yes & Yes & Numerical \\
\hline
PAG 2021    & Crest and action    & Yes & No  & No  & Numerical \\
            & conservation        &     &     &     &           \\
\hline
This paper  & Crest and action    & Yes & Yes & No  & Analytical \\
            & conservation        &     &     &     & and numerical \\
\hline
\end{tabular}
\caption{
  Summary of prior
  \citep{longuet1960changes,phillips1981dispersion,longuet1987propagation,henyey1988energy,zhang1990evolution,peureux2021unsteady}
  and present (this paper) approaches to calculating the modulation of short
  waves by long waves.
}
\label{table:modulation_literature_summary}
\end{center}
\end{table}

This paper first presents the governing equations and then the analytical
solutions for the modulation of short waves by long waves and compares them to
the original solution by \citet{longuet1960changes}.
Then, we compare the analytical solutions to numerical solutions of the full
wave conservation equations in section \ref{section:numerical_solutions}.
In section \ref{section:scale_analysis} we perform a scale analysis of the
analytical solutions to quantify the relative importance of the terms in the
full equation set.
Finally, we summarize the results and discuss their implications in section
\ref{section:conclusions}.

\section{Governing equations}
\label{section:governing_equations}

We consider inviscid, irrotational, and incompressible flow.
In these conditions the linear (sine-like), deep-water, surface gravity waves
obey the dispersion relation:

\begin{equation}
\label{eq:dispersion}
\omega = \sqrt{gk} + k U
\end{equation}
where $\omega$ is the angular frequency, $g$ is the gravitational acceleration,
$k$ is the wavenumber, and $U$ is the mean advective current in the direction
of the wave propagation.
A current in the direction of the waves increases its absolute frequency,
while the wavenumber remains unchanged.
An important limitation here is that $U$ must be slowly varying on the scales of
the wave period \citep{bretherton1968wavetrains}.
In the context of short waves riding on long waves, the mean advective current
is the horizontal near-surface orbital velocity of the long wave.
The evolution of the short-wave wavenumber is described by the conservation of
wave crests \citep{phillips1981dispersion}:

\begin{equation}
\label{eq:wave_crests}
\dfrac{\partial k}{\partial t}
+ \dfrac{\partial \omega}{\partial x}
= 0
\end{equation}
where $t$ and $x$ are time and space, respectively.
This relation states that the wavenumber must change when the frequency varies
in space.

Equations (\ref{eq:dispersion}) and (\ref{eq:wave_crests}) only describe the
evolution of the short-wave wavenumber.
To describe the evolution of the wave amplitude, we use the conservation of wave
action in absence of sources (e.g. due to wind) and sinks (e.g. due to whitecapping):

\begin{equation}
\label{eq:wave_action}
\dfrac{\partial N}{\partial t}
+ \dfrac{\partial}{\partial x} \left[\left(C_g + U\right)N\right]
= 0
\end{equation}
where $N$ is the action density, and $C_g = \partial \omega / \partial k$ is the
group speed.
In deep water, the group speed is half the phase speed, so $C_g = 1/2\sqrt{g/k}$.
Although the absence of wave growth and dissipation is not generally realistic
in the open ocean, it is a helpful approximation that allows us to isolate the
hydrodynamic modulation effects solely due to the motion of the long waves.

Short waves that ride on the surface of the longer waves move in an accelerated
reference frame and thus experience effective gravitational acceleration that
changes in space and time \citep{phillips1981dispersion,longuet1986eulerian,longuet1987propagation}.
From Eqs. (\ref{eq:dispersion}) and (\ref{eq:wave_action}) we obtain:

\begin{equation}
\label{eq:wavenumber}
\dfrac{\partial k}{\partial t}
+ \left(C_g + U\right) \dfrac{\partial k}{\partial x}
+ k \dfrac{\partial U}{\partial x}
+ \dfrac{1}{2} \sqrt{\dfrac{k}{g}} \dfrac{\partial g}{\partial x}
= 0
\end{equation}
From left to right, the terms in this equation represent the change in wavenumber
due to the propagation and advection by ambient current $U$, the convergence of
the ambient current, and the spatial inhomogeneity of the gravitational acceleration.
In a non-accelerated reference frame (i.e. in absence of longer waves), the last
term vanishes.
Otherwise, this term is necessary to conserve the wave crests.

Similar to Eq. \ref{eq:wavenumber}, we expand Eq. \ref{eq:wave_action} to obtain:

\begin{equation}
\label{eq:wave_action2}
\dfrac{\partial N}{\partial t}
+ \left(C_g + U\right) \dfrac{\partial N}{\partial x}
+ N \dfrac{\partial U}{\partial x}
+ N \dfrac{\partial C_g}{\partial x}
= 0
\end{equation}
This equation is similar to Eq. \ref{eq:wavenumber}, except for the last term
which here represents the inhomogeneity of the group speed.
As \citet{peureux2021unsteady} explained, this term is responsible for unstable
growth of wave action in infinite long-wave trains.
Eqs. \ref{eq:wavenumber} and \ref{eq:wave_action2} are the governing equations
that we approximate and solve analytically in section \ref{section:analytical_solutions},
and solve numerically in their full form in section \ref{section:numerical_solutions}.

For the ambient forcing, we consider a train of monochromatic long wave
defined, to the first order in steepness, by the elevation $\eta_L = a_L cos{\psi}$,
where $a_L$ is the long-wave amplitude, $\psi = k_L x - \omega_L t$ is the
long-wave phase, and $\omega_L$ its angular frequency.
At this time we define the long-wave steepness $\varepsilon_L = a_L k_L$ which
we will use throughout the rest of the paper as a scaling parameter, as is common
in the literature.
The long wave induces horizontal and vertical orbital velocities whose values
at the surface $\eta_L$ are:

\begin{equation}
\label{eq:U_L}
U = a_L \omega_L e^{\varepsilon_L \cos{\psi}} \cos{\psi}
\end{equation}

\begin{equation}
\label{eq:w_L}
W = a_L \omega_L e^{\varepsilon_L \cos{\psi}} \sin{\psi}
\end{equation}

Eqs. \ref{eq:dispersion}-\ref{eq:w_L} is the full system of equations that we
use in this paper.
Note that here we derive the solutions from a different starting set of
equations than that of \citet{longuet1960changes}.
Rather than starting from the conservation of crests and wave action, as we do
here, they derived their solution from the linear superposition of the velocity
potential of two waves.
It appears that starting from the conservation of crests and wave action allows
for a simpler and more intuitive derivation.

\section{Analytical solutions}
\label{section:analytical_solutions}

We now describe the analytical solutions for the short-wave wavenumber,
effective gravitational acceleration, amplitude, steepness, intrinsic frequency,
and intrinsic phase speed in presence of long waves.

\subsection{Wavenumber modulation}
\label{subsection:wavenumber_modulation}

Solving analytically for the wavenumber requires dropping spatial gradients of
$k$ in Eq. \ref{eq:wavenumber} and assuming homogeneous $g$:

\begin{equation}
\label{eq:wavenumber2}
\dfrac{\partial k}{\partial t}
= - k \dfrac{\partial U}{\partial x}
\end{equation}
Although \citet{peureux2021unsteady} pointed out that the solution by L-HS
requires assuming homogeneous group speed of the short-wave field, in fact,
it also requires assuming no horizontal propagation and advection of the short
waves.
Combining Eqs. \ref{eq:U_L} and \ref{eq:wavenumber2}, and integrating it in time
yields:

\begin{equation}
\label{eq:k_short_exact}
\widetilde{k}(\psi) = k e^{\varepsilon_L \cos{\psi} e^{\varepsilon_L \cos{\psi}}}
\end{equation}

Notice that the Taylor expansion of Eq. \ref{eq:k_short_exact} to the first order
recovers the original solution by L-HS:

\begin{equation}
\label{eq:k_short_lhs}
\widetilde{k}(\psi) = k (1 + \varepsilon_L \cos{\psi})
\end{equation}

The L-HS solution requires two approximations in their solution:
First, evaluating the long-wave velocity at $z = 0$
rather than $z = \eta_L$; and second, evaluating the solution to the first order
of the Taylor expansion series.
Both of these approximations cause an underestimate of the short-wave
modulation magnitude (Fig. \ref{fig:wavenumber_modulation}).
The original L-HS solution is a cosine wave with an amplitude equal
to the long-wave steepness.
The other two solutions both introduce an increase in wave modulation
on the crest and a decrease in the trough.
At the crest, the relative modulation predicted by the exact L-HS solution
is 10.7\% larger than that predicted by the 1st order approximate L-HS
solution, while the solution that evaluates the long-wave orbital velocity
at the water surface rather than $z = 0$ is 38.4\% larger than that of the
1$^{st}$ order approximate L-HS solution.

\begin{figure}
\centering
\includegraphics[width=0.8\textwidth]{figures/fig_analytical_solutions_ak0.1.png}
\caption{
  Modulation of short-wave (a) wavenumber, (b) gravitational acceleration,
  (c) amplitude, (d) steepness, (e) intrinsic frequency, and (d) intrinsic phase
  speed as function of long-wave phase for $\varepsilon_L = 0.1$, based on
  analytical solutions by \citet{longuet1960changes} (L-HS 1960, black)
  and this paper (blue).
  Long-wave crest and trough are located at $\psi = 0$ and $\psi = \pi$,
  respectively.
}
\label{fig:analytical_solutions_ak0.1}
\end{figure}

\subsection{Gravity modulation}
\label{subsection:gravity_modulation}

Long waves induce orbital accelerations, so the short waves that ride on their
surface move in an accelerated reference frame and experience effective
gravitational acceleration that is lower at the crests and higher in the troughs
\citep{longuet1986eulerian,longuet1987propagation}.
Further, a smaller but non-negligible effect is that of the centripetal force
due to the curvature of the long wave
\citep{phillips1981dispersion,zhang1990evolution}.
These effects are completely described by projecting the gravitational
acceleration from the coordinate that is perpendicular to the mean water surface
($z=0$) to that which is perpendicular to the long-wave surface ($z=\eta$),
and likewise for the long-wave orbital accelerations:

\begin{equation}
\label{eq:gravity_modulation_general}
\widetilde{g}
  = g \cos{\alpha} 
  + \dfrac{\partial W_{z=\eta}}{\partial t} \cos{\alpha}
  + \dfrac{\partial U_{z=\eta}}{\partial t} \sin{\alpha}
\end{equation}

\begin{equation}
\label{eq:local_slope}
\alpha = \tan^{-1}{\dfrac{\partial \eta}{\partial x}}
\end{equation}
where $\alpha$ is the local slope of the long-wave surface.
For a linear wave, the effective gravity is:

\begin{equation}
\label{eq:gravity_modulation_linear}
\widetilde{g}
  = g 
  \frac{
    1 - \varepsilon_L \cos{\psi} e^{\varepsilon_L \cos{\psi}}
    \left[ 1 + \left(\varepsilon_L \sin{\psi}\right)^2 \right]
  }
  {\sqrt{\left(\varepsilon_L \sin{\psi}\right)^2 + 1}}
\end{equation}

Curvilinear effects on effective gravity can be neglected by setting $\alpha = 0$.

The first-order term in $a_L k_L$ is in phase with the long-wave elevation,
so the apparent gravity of the short waves is increased at the long-wave crests
and decreased in the troughs.
The second-order term in $a_L k_L$ is positive-definite and phase-shifted by
$\pi/2$ relative to the crests.
The second-order term thus increases the apparent gravity on the front and rear
faces of the long wave (Fig. \ref{fig:gravity_modulation}).

We can also derive the short-wave gravitational acceleration on the surface of
a non-linear Stokes wave with the elevation (to the third order in $\varepsilon_L$):

\begin{equation}
\label{eq:eta_stokes}
\eta_{St} = a_L \left[
  \cos{\psi} +
  \dfrac{1}{2} \varepsilon_L \cos{2\psi} +
  \varepsilon_L^2 \left( \dfrac{3}{8} \cos{3\psi} - \dfrac{1}{16} \cos{\psi} \right)
\right]
\end{equation}
Without considering the curvilinear effects in Eq. \ref{eq:gravity_modulation_general},,
the gravitational acceleration at the surface of a Stokes wave is, to the
4$^{th}$ order:

\begin{equation}
\label{eq:gravity_modulation_stokes}
\widetilde{g}(\psi) =
g \left\{
  1 - \varepsilon_L e^{k \eta_{St}}
  \left[ \cos{\psi} -
    \varepsilon_L \sin{\psi} \left(
      \sin{\psi}
      + \varepsilon_L \sin{2\psi}
      - \dfrac{1}{16} \varepsilon_L^2 \sin{\psi}
      + \dfrac{9}{8} \varepsilon_L^2 \sin{3\psi}
    \right)
  \right]
\right\}
\end{equation}
The equivalent expression that includes the curvilinear effects can be derived
by evaluating orbital accelerations and local slope $\alpha$ at $z = \eta_{St}$,
and inserting them into Eq. \ref{eq:gravity_modulation_general}:

\begin{equation}
\label{eq:gravity_modulation_general_stokes}
\widetilde{g}
  = g \cos{\alpha} 
  + \dfrac{\partial W_{z=\eta_{St}}}{\partial t} \cos{\alpha_{St}}
  + \dfrac{\partial U_{z=\eta_{St}}}{\partial t} \sin{\alpha_{St}}
\end{equation}

\begin{equation}
\label{eq:local_slope_stokes}
\alpha_{St} = \tan^{-1}{\dfrac{\partial \eta_{St}}{\partial x}}
\end{equation}

\begin{figure}
\centering
\includegraphics[width=0.6\textwidth]{figures/fig_effective_gravities.png}
\caption{
  (a) Analytical solutions for the effective gravitational acceleration
  modulation by long waves, as function of long-wave phase, for $\varepsilon_L = 0.2$.
  Black is for linear wave evaluated at $z=0$,
  blue and green are for linear and Stokes waves, respectively, evaluated at $z=\eta$ and without curvature effects,
  and orange and green are for linear and Stokes waves, respectively, evaluated at $z=\eta$ and with curvature effects.
  (b) As in (a) but normalized by the modulation of the linear wave evaluated at $z=0$.
}
\label{fig:effective_gravities}
\end{figure}

\begin{figure}
\centering
\includegraphics[width=0.8\textwidth]{figures/fig_effective_gravities_mean_max_min.png}
\caption{
  As Fig. \ref{fig:effective_gravities} but showing the mean (solid),
  maximum (dashed), and minimum (dotted) short-wave gravity modulation as
  function of long-wave steepness $\varepsilon_L$.
}
\label{fig:effective_gravities_mean_max_min}
\end{figure}

\subsection{Amplitude and steepness modulation}
\label{subsection:amplitude_modulation}

The modulation of short-wave amplitude can be derived in the same way by
dropping the non-linear terms in the wave action balance equation
(Eq. \ref{eq:wave_action2}) and integrating wave action in time:

\begin{equation}
\label{eq:wave_action_modulation}
\widetilde{N}(\psi) = N e^{\varepsilon_L \cos{\psi} e^{\varepsilon_L \cos{\psi}}}
\end{equation}

Since wave action is energy divided by the intrinsic frequency $\sigma$,
and energy scales with $ga^2$, the modulation of the amplitude is readily
related to the modulations of gravity, wavenumber, and action:

\begin{equation}
\label{eq:wave_amplitude_modulation}
\dfrac{\widetilde{a}}{a} = \sqrt{
  \dfrac{g}{\widetilde{g}}
  \dfrac{\widetilde{\sigma}}{\sigma}
  \dfrac{\widetilde{N}}{N}}
=
  \left( \dfrac{g}{\widetilde{g}} \right)^{0.25}
  \left( \dfrac{\widetilde{k}}{k} \right)^{0.25}
  \left( \dfrac{\widetilde{N}}{N} \right)^{0.5}
\end{equation}
To the 1$^{st}$ order in $\varepsilon_L$, and using Eqs. \ref{eq:k_short_exact}
and \ref{eq:wave_action_modulation}, we get:

\begin{equation}
\label{eq:wave_amplitude_modulation_order1}
\dfrac{\widetilde{a}}{a} = 
  \left( \dfrac{g}{\widetilde{g}} \right)^{0.25}
  \left( 1 + \varepsilon_L \right)^{0.75}
\end{equation}

\begin{figure}
\centering
\includegraphics[width=0.8\textwidth]{figures/fig_analytical_solutions_ak0.2.png}
\caption{As Fig. \ref{fig:analytical_solutions_ak0.1} but for $\varepsilon_L = 0.2$.}
\label{fig:analytical_solutions_ak0.2}
\end{figure}

\begin{figure}
\centering
\includegraphics[width=0.8\textwidth]{figures/fig_analytical_solutions_ak0.4.png}
\caption{As Fig. \ref{fig:analytical_solutions_ak0.1} but for $\varepsilon_L = 0.4$.}
\label{fig:analytical_solutions_ak0.4}
\end{figure}

\section{Numerical solutions}
\label{section:numerical_solutions}

We now proceed to integrate the full numerical solutions for the short-wave
wavenumber and action density in presence of long waves, and in absence of
sources or sinks of wave energy.
Equations (\ref{eq:wavenumber})-(\ref{eq:wave_action2}) are discretized using
2$^{nd}$  order central finite difference in space and integrated in time using the
4$^{th}$ order Runge-Kutta method \citep{butcher1996runge}.
The space is divided into 100 grid points.
This is effectively the same equation set and numerical configuration as that of
\citet{peureux2021unsteady}, except that, like \citet{longuet1960changes}, their
long-wave orbital velocities are evaluated at $z=0$ instead of $z=\eta_L$.
The key difference is that evaluating orbital velocities at $z=0$ neglects the
Stokes drift of short-wave groups.
Although not significantly consequential for the modulation results here, there
is no good reason to exclude it given that its implementation is trivial.

Another difference is the choice of the numerical scheme for spatial differences,
which in their case is the MUSCL4 scheme by Kurganov and Tadmor (2000 TODO).
Although the 2$^{nd}$ order central finite difference is not appropriate for many
numerical problems, in our case it is sufficient because it is conservative and
the fields that we compute derivatives of are smooth and continuous.
We show in the next section that our numerical results in infinite long-wave
trains are qualitatively equivalent to those of \citet{peureux2021unsteady}.

Finally, the numerical equations here are integrated in a fixed reference frame,
rather than one that moves with the long wave phase speed.
Either approach produces qualitatively equivalent results, but we find that
a fixed reference frame allows for more intuitive interpretation of the results.
However, this difference is important to keep in mind when comparing the results
of \citet{peureux2021unsteady} and those presented here--in their Fig. 1 the
long-wave phase is fixed and the short waves are moving leftward
with the speed of $C_{pL} - C_g - U$ (neglecting any group speed inhomogeneity),
whereas in the figures that follow, the long wave is moving rightward with its
phase speed $C_{pL}$ and the short waves (that is, their energy) are moving
rightward with the speed of $C_g + U$ (neglecting any group wave inhomogeneity).

\subsection{Comparison with analytical solutions}
\label{subsection:comparison_with_analytical_solutions}

\begin{figure}
\centering
\includegraphics[width=\textwidth]{figures/numerical_vs_analytical_solutions_with_phase.png}
\caption{
  Comparison of numerical solutions of wavenumber, amplitude, and steepness modulation
  with their analytical solutions, for $\varepsilon_L$ = 0.1, 0.2, and 0.4.
}
\label{fig:numerical_solutions}
\end{figure}

\begin{figure}
\centering
\includegraphics[width=0.6\textwidth]{figures/numerical_vs_analytical_solutions_k.png}
\caption{}
\label{fig:numerical_vs_analytical_k}
\end{figure}

\begin{figure}
\centering
\includegraphics[width=0.6\textwidth]{figures/numerical_vs_analytical_solutions_a.png}
\caption{}
\label{fig:numerical_vs_analytical_a}
\end{figure}

\begin{figure}
\centering
\includegraphics[width=0.6\textwidth]{figures/numerical_vs_analytical_solutions_ak.png}
\caption{}
\label{fig:numerical_vs_analytical_ak}
\end{figure}

\subsection{Unstable growth in Pereux et al. (2021)}
\label{subsection:unstable_growth}

The key result of \citet{peureux2021unsteady} is that of unsteady steepening of
short waves when a homogeneous field of short waves is forced by an infinite
long-wave train.
The steepening is driven by the progressive increase in the action of short
waves (as opposed to their wavenumber), and is centered on the group of short
waves that begins its journey in the trough of the long wave.
This is because the unmodulated short waves that begin in the trough enter into
the area where the surface velocities are converging.
Inversely, the short waves that begin around the crest of the long-wave will
enter into the area of surface divergence and will thus have their wave action
and wavenumber, decreased.
We reproduce this behavior in Fig. \ref{fig:modulation_3panel_infinite}, which
shows the evolution of the short-wave action, wavenumber, and steepness
modulation (here defined as the change of a quantity relative to its initial value).
After 10 long-wave periods, the short-wave action modulation $\widetilde{N}/N$
is approximately doubled, which almost exactly corresponds to the unsteady growth
reported by \citet{peureux2021unsteady} in their Fig. 1.
Another important property of this solution is that the short-wave group that
experiences peak amplification (dampening) are not locked-in to the long-wave
crests (troughs), as they are in the L-HS solution.

\begin{figure}
\centering
\includegraphics[width=\textwidth]{figures/fig_modulation_3panel_infinite_wave_train.png}
\caption{
  Change in the short-wave action (left), wavenumber (middle), and steepness (right)
  relative to their initial values as function of initial long-wave phase and time.
  $k_L = 1$, $k = 10$, $\varepsilon_L = 0.1$.
}
\label{fig:modulation_3panel_infinite}
\end{figure}

Why does this unsteady growth of short-wave action occur?
As \citet{peureux2021unsteady} explain, "the additional change in advection
velocity due to the modulation of the short wave group speed gives an additional
amplification."
That is, it is the inhomogeneity term $N \dfrac{\partial C_g}{\partial x}$ that
is responsible for the instability.
However, the instability only occurs if the short waves are initialized as a
uniform field.
If they are initialized instead as a periodic function of the long-wave phase
such that their wavenumber (and optionally, action) is higher on the crests
and lower in the troughs, akin to the L-HS solution, the instability does not
occur.

\begin{figure}
\centering
\includegraphics[width=\textwidth]{figures/fig_modulation_3panel_linear_ramp.png}
\caption{
  Same as Fig. \ref{fig:modulation_3panel_infinite} but with a linear ramp
  applied to $a_L$ during the initial five long-wave periods.
}
\label{fig:modulation_3panel_ramp}
\end{figure}

Why is it, then, that the solution is so different depending on the short-wave
initial conditions?
\citet{peureux2021unsteady} did not provide an explanation for this behavior.
At first, it may seem that there is something special about the uniform initial
conditions that makes its solutions grow unsteadily and indefinitely.
\citet{peureux2021unsteady} describe this scenario as "...the sudden appearance
of a long wave perturbation in the middle of a homogeneous short wave field".
The sudden appearance of long waves may be a clue.
The exact numerical simulation as shown in Fig. \ref{fig:modulation_3panel_infinite},
except for a linear ramp applied to $a_L$ (and consequently, $U$ and $W$ as well)
for the first five periods reveals that the unsteady growth no longer occurs
(Fig. \ref{fig:modulation_3panel_ramp}).

\begin{figure}
  \centering
  \includegraphics[width=0.6\textwidth]{figures/fig_modulation_ramp_timeseries.png}
  \caption{
    Maximum short-wave steepness modulation as function of time in case of
    infinite long-wave train case (blue) and the same case but with a linear
    ramp during the initial five long-wave periods (orange). The dashed black
    line corresponds to the short-wave steepness of 0.4 at which most waves are
    expected to break.
  }
  \label{fig:unsteady_growth_timeseries}
\end{figure}

\subsection{Short-wave modulation by long-wave groups}
\label{subsection:wave_groups}

\begin{figure}
\centering
\includegraphics[width=\textwidth]{figures/fig_modulation_3panel_wave_group.png}
\caption{
  Same as Fig. \ref{fig:modulation_3panel_infinite} but for a long-wave group,
  causing the long-wave amplitude to gradually increase and peak at $a_L = 0.1$
  after five long-wave periods, and then conversely decay back to a calm sea state.
}
\label{fig:modulation_3panel_groups}
\end{figure}

\section{Scale analysis}
\label{section:scale_analysis}

Now that we showed that the analytical solution is consistent with full numerical
solutions for $\varepsilon_L < 0.25$, we can use analytical solutions for scale
analysis to quantify the relative contribution of tendencies in the full equation set.

TODO

\section{Conclusions}
\label{section:conclusions}

In this paper we revisited the problem of short-wave modulation by long waves
from a linear wave theory perspective.
Using the linearized wave crest and action conservation equations, coupled with
the nonlinear gravity modulation for short waves that includes the the curvature
effects of the long wave, we derived new analytical solutions for the modulation
of short wave wavenumber, amplitude, steepness, intrinsic frequency, and phase
speed.
The latter two have not been previously examined in the context of short-wave
modulation, but are important for correctly calculating the wind input into
short waves, which is proportional to the wind speed relative to the wave
celerity.
The approximate steady solutions yield higher wavenumber and amplitude
modulation than that of \citet{longuet1960changes}, and somewhat lower than 
the numerical solutions of \citet{longuet1987propagation} and
\citet{zhang1990evolution}.
This is mainly due to the analytical solution requiring the crest and action
conservation equations to be linearized, and the nonlinearity arising only due
to the evaluation of short waves on the surface of the long wave (as opposed to
the mean water level) and due to nonlinear effective gravity.

The numerical solutions of the full wave crest and action conservation equations
are consistent with the analytical solutions for $\varepsilon_L < 0.2$.
The simulations also reveal that the unsteady growth of short-wave steepness
reported by \citet{peureux2021unsteady} is due to the sudden appearance of
long waves in a homogeneous short-wave field, a scenario that is unlikely to
occur in nature.
A more likely conditions, such as that of a long-wave groups, are only mildly
destabilizing for the short waves.
These results suggest that for mildly-to-moderately steep short waves,
the analytical steady solutions presented here are sufficient to describe the
modulation of short waves by long waves.

\backsection[Acknowledgements]{
Thanks to Nathan Laxague, Fabrice Ardhuin, Nick Pizzo, and Peisen Tan for the
helpful discussions.
}

\backsection[Funding]{
  Milan Curcic has been partially supported by the National Science Foundation
  under grant number AGS-1745384.
}

\backsection[Declaration of interests]{The author reports no conflict of interest.}

\backsection[Data availability statement]{
  The data and code to generate the figures and run the numerical simulations
  are available at [TODO].
  The numerical model is available as a standalone package at [TODO].
  %See JFM's \href{https://www.cambridge.org/core/journals/journal-of-fluid-mechanics/information/journal-policies/research-transparency}{research transparency policy} for more information
}

\backsection[Author ORCIDs]{M. Curcic, https://orcid.org/0000-0002-8822-7749}

%\backsection[Author contributions]{Authors may include details of the contributions made by each author to the manuscript}

\bibliographystyle{jfm}
\bibliography{references}

\end{document}
